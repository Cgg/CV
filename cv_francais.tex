\documentclass{article}

\usepackage[francais]{babel}
\usepackage[utf8]{inputenc}
\usepackage[T1]{fontenc}
%\usepackage{charter}
%\usepackage{lmodern}
%\usepackage{utopia}
%\usepackage{kpfonts}
%\usepackage{yfonts}
%\usepackage{venturis2}
%\usepackage{frcursive}
%\usepackage[defaultsans]{comfortaa}
%\usepackage{arev}
\usepackage[scaled]{berasans}
\renewcommand*\familydefault{\sfdefault}  %% Only if the base font of the document is to be sans serif
%\renewcommand*\oldstylenums[1]{{\fontfamily{fxlj}\selectfont #1}}
\usepackage{geometry}
\usepackage[NoDate,LabelsAligned]{currvita}
\setlength{\cvlabelskip}{%
  5pt%
}%
\setlength{\cvlabelsep}{0.5em}
\geometry{%
	a4paper,
	body={180mm,275mm},
	%left=30mm,top=40mm,
	left=15mm,top=10mm,	
	headheight=0mm,headsep=0mm,
	marginparsep=4mm,
	marginparwidth=27mm}

\newcommand{\alin}{\hskip 6mm}
\newcommand{\saut}{\vskip 2mm}

\newcommand{\cvitem}[1]{%
\raggedright{
\vskip 1mm
\makebox[\textwidth]{\rule{11mm}{0.4pt} \hskip 2mm {\Large\bf\sc #1} \hskip 2mm \leaders \hrule \hfill}
}
}


\begin{document}
\thispagestyle{empty}
\raggedright{%
\begin{minipage}{8cm}%
{\LARGE \bf Clément GEIGER}%
\vskip 2mm
176 rue Philippe de Commynes\\%
45~160 Olivet\\ %
France\\ %
{\verb clement.geiger@gmail.com }\\
\textbf{+33~(0)6~29~82~60~55}
\end{minipage}}\hfill
\begin{minipage}{8cm}
\begin{flushright}
21 ans\\
Nationalité française\\
\textbf{Mobilité internationale}
\end{flushright}
\end{minipage}

\vskip 5mm
{\large\textit{Objet}: Recherche d'un projet de fin d'étude validant mon
diplôme d'ingénieur informatique à \textbf{INSA de Lyon}

\begin{cv}{}% Élève ingénieur en informatique }
\vskip 4mm
\cvitem{Formation}
    \begin{cvlist}{}
        \item[2012]             Échange Erasmus à KTH (Stockholm, Suède)
        \item[2009 - 2012] 	Étudiant à l'INSA de Lyon, département informatique
        \item[2007 - 2009] 	1\up{er} cycle à l'INSA de Lyon : formation scientifique générale : mathématiques, physique, chimie
        \item[2007]		Baccalauréat scientifique, mention Très Bien
    \end{cvlist}
    \saut

\cvitem{Expériences professionnelles}
    \begin{cvlist}{}
        \item[2011]     Stage d'été à \textbf{Schlumberger} (Abingdon,
        Royaume-Uni)
        \item[(4 mois)] \alin Experiments on multiple constraints in multi-segmented wells
        \item           \alin \textit{Technologies}: MS Visual Studio, C++, Perforce, linear algebra
    \saut
        \item[2010]     Stage d'été à \textbf{Philips Applied Technologies}
        (Eindhoven, Pays-Bas)
        \item[(3 mois)] \alin Conception and integration of two extension for an audio demonstrator software
        \item           \alin \textit{Technologies}: MS Visual Studio, C++, multithread programming
    \saut
	\item[2009]	Stage d'été à \textbf{Thales Air Defence}
        (Fleury-les-Aubrais, France)
	\item[(1 mois)]  \alin Design and construction of two graphical
	configuration generators for existing tools
    \end{cvlist}
    \saut

\cvitem{Compétences}
    \begin{cvlist}{}
	\item[Programmation]	Analyse : UML, USDP
	\item			Languages objets: \textbf{C++}, Java
        \item                   Scripting : Python, Javascript, Lua, Shell, Matlab
	\item			Frameworks : STL, \textbf{Qt}
	\item			Bas niveau, embarqué : C, C sur
        Microchip Pic, assembleur Motorola 68k, VxWorks OS, $\mu$C/OSII
    \saut
        \item[Connaissances]    Intelligence artificielle, traitement
        d'images, théorie du signal, réseau
    \saut
	\item[Transverse skills]Travail collaboratif : \textbf{Git}, Subversion, Perforce
	\item			Utilisation courante de \LaTeX
	\item			Gestion de projet, assurance qualité
    \saut
	\item[Langues]	        \textbf{Anglais courant et technique} - score TOEIC : \textbf{980}/990
	\item			français : langue maternelle ; allemand lu et parlé
    \end{cvlist}
    \saut

\cvitem{Centres d'intérêt}
\vskip 1mm
	\textbf{Photographie}: pratique argentique et numérique
        \textbf{Sport}: Aviron, escalade (voie et bloc)
        \textbf{Music}: HiFi, jazz et musique électronique
        \textbf{IT}: Linux et logiciel libre
\end{cv}
\vfill
\end{document}
